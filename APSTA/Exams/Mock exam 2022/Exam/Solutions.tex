% Options for packages loaded elsewhere
\PassOptionsToPackage{unicode}{hyperref}
\PassOptionsToPackage{hyphens}{url}
%
\documentclass[
]{article}
\usepackage{amsmath,amssymb}
\usepackage{lmodern}
\usepackage{iftex}
\ifPDFTeX
  \usepackage[T1]{fontenc}
  \usepackage[utf8]{inputenc}
  \usepackage{textcomp} % provide euro and other symbols
\else % if luatex or xetex
  \usepackage{unicode-math}
  \defaultfontfeatures{Scale=MatchLowercase}
  \defaultfontfeatures[\rmfamily]{Ligatures=TeX,Scale=1}
\fi
% Use upquote if available, for straight quotes in verbatim environments
\IfFileExists{upquote.sty}{\usepackage{upquote}}{}
\IfFileExists{microtype.sty}{% use microtype if available
  \usepackage[]{microtype}
  \UseMicrotypeSet[protrusion]{basicmath} % disable protrusion for tt fonts
}{}
\makeatletter
\@ifundefined{KOMAClassName}{% if non-KOMA class
  \IfFileExists{parskip.sty}{%
    \usepackage{parskip}
  }{% else
    \setlength{\parindent}{0pt}
    \setlength{\parskip}{6pt plus 2pt minus 1pt}}
}{% if KOMA class
  \KOMAoptions{parskip=half}}
\makeatother
\usepackage{xcolor}
\usepackage[margin=1in]{geometry}
\usepackage{color}
\usepackage{fancyvrb}
\newcommand{\VerbBar}{|}
\newcommand{\VERB}{\Verb[commandchars=\\\{\}]}
\DefineVerbatimEnvironment{Highlighting}{Verbatim}{commandchars=\\\{\}}
% Add ',fontsize=\small' for more characters per line
\usepackage{framed}
\definecolor{shadecolor}{RGB}{248,248,248}
\newenvironment{Shaded}{\begin{snugshade}}{\end{snugshade}}
\newcommand{\AlertTok}[1]{\textcolor[rgb]{0.94,0.16,0.16}{#1}}
\newcommand{\AnnotationTok}[1]{\textcolor[rgb]{0.56,0.35,0.01}{\textbf{\textit{#1}}}}
\newcommand{\AttributeTok}[1]{\textcolor[rgb]{0.77,0.63,0.00}{#1}}
\newcommand{\BaseNTok}[1]{\textcolor[rgb]{0.00,0.00,0.81}{#1}}
\newcommand{\BuiltInTok}[1]{#1}
\newcommand{\CharTok}[1]{\textcolor[rgb]{0.31,0.60,0.02}{#1}}
\newcommand{\CommentTok}[1]{\textcolor[rgb]{0.56,0.35,0.01}{\textit{#1}}}
\newcommand{\CommentVarTok}[1]{\textcolor[rgb]{0.56,0.35,0.01}{\textbf{\textit{#1}}}}
\newcommand{\ConstantTok}[1]{\textcolor[rgb]{0.00,0.00,0.00}{#1}}
\newcommand{\ControlFlowTok}[1]{\textcolor[rgb]{0.13,0.29,0.53}{\textbf{#1}}}
\newcommand{\DataTypeTok}[1]{\textcolor[rgb]{0.13,0.29,0.53}{#1}}
\newcommand{\DecValTok}[1]{\textcolor[rgb]{0.00,0.00,0.81}{#1}}
\newcommand{\DocumentationTok}[1]{\textcolor[rgb]{0.56,0.35,0.01}{\textbf{\textit{#1}}}}
\newcommand{\ErrorTok}[1]{\textcolor[rgb]{0.64,0.00,0.00}{\textbf{#1}}}
\newcommand{\ExtensionTok}[1]{#1}
\newcommand{\FloatTok}[1]{\textcolor[rgb]{0.00,0.00,0.81}{#1}}
\newcommand{\FunctionTok}[1]{\textcolor[rgb]{0.00,0.00,0.00}{#1}}
\newcommand{\ImportTok}[1]{#1}
\newcommand{\InformationTok}[1]{\textcolor[rgb]{0.56,0.35,0.01}{\textbf{\textit{#1}}}}
\newcommand{\KeywordTok}[1]{\textcolor[rgb]{0.13,0.29,0.53}{\textbf{#1}}}
\newcommand{\NormalTok}[1]{#1}
\newcommand{\OperatorTok}[1]{\textcolor[rgb]{0.81,0.36,0.00}{\textbf{#1}}}
\newcommand{\OtherTok}[1]{\textcolor[rgb]{0.56,0.35,0.01}{#1}}
\newcommand{\PreprocessorTok}[1]{\textcolor[rgb]{0.56,0.35,0.01}{\textit{#1}}}
\newcommand{\RegionMarkerTok}[1]{#1}
\newcommand{\SpecialCharTok}[1]{\textcolor[rgb]{0.00,0.00,0.00}{#1}}
\newcommand{\SpecialStringTok}[1]{\textcolor[rgb]{0.31,0.60,0.02}{#1}}
\newcommand{\StringTok}[1]{\textcolor[rgb]{0.31,0.60,0.02}{#1}}
\newcommand{\VariableTok}[1]{\textcolor[rgb]{0.00,0.00,0.00}{#1}}
\newcommand{\VerbatimStringTok}[1]{\textcolor[rgb]{0.31,0.60,0.02}{#1}}
\newcommand{\WarningTok}[1]{\textcolor[rgb]{0.56,0.35,0.01}{\textbf{\textit{#1}}}}
\usepackage{graphicx}
\makeatletter
\def\maxwidth{\ifdim\Gin@nat@width>\linewidth\linewidth\else\Gin@nat@width\fi}
\def\maxheight{\ifdim\Gin@nat@height>\textheight\textheight\else\Gin@nat@height\fi}
\makeatother
% Scale images if necessary, so that they will not overflow the page
% margins by default, and it is still possible to overwrite the defaults
% using explicit options in \includegraphics[width, height, ...]{}
\setkeys{Gin}{width=\maxwidth,height=\maxheight,keepaspectratio}
% Set default figure placement to htbp
\makeatletter
\def\fps@figure{htbp}
\makeatother
\setlength{\emergencystretch}{3em} % prevent overfull lines
\providecommand{\tightlist}{%
  \setlength{\itemsep}{0pt}\setlength{\parskip}{0pt}}
\setcounter{secnumdepth}{-\maxdimen} % remove section numbering
\ifLuaTeX
  \usepackage{selnolig}  % disable illegal ligatures
\fi
\IfFileExists{bookmark.sty}{\usepackage{bookmark}}{\usepackage{hyperref}}
\IfFileExists{xurl.sty}{\usepackage{xurl}}{} % add URL line breaks if available
\urlstyle{same} % disable monospaced font for URLs
\hypersetup{
  pdftitle={APSTA 2022 Mock Exam solutions},
  hidelinks,
  pdfcreator={LaTeX via pandoc}}

\title{APSTA 2022 Mock Exam solutions}
\author{}
\date{\vspace{-2.5em}}

\begin{document}
\maketitle

\hypertarget{probability-theory}{%
\section{1. Probability Theory}\label{probability-theory}}

Consider an experiment where you flip a fair coin four times.\\
(a) Define a natural sample space \(\Omega\) for this experiment.\\
\emph{Solution}:\\
\[
\begin{aligned}
\Omega=&\\
&\{(H,H,H,H), (H,H,H,T), (H,H,T,H), (H,H,T,T), \\
&(H,T,H,H), (H,T,H,T), (H,T,T,H), (H,T,T,T),  \\
&(T,H,H,H), (T,H,H,T), (T,H,T,H), (T,H,T,T),  \\
&(T,T,H,H), (T,T,H,T), (T,T,T,H), (T,T,T,T)\}
\end{aligned}
\]

\begin{enumerate}
\def\labelenumi{(\alph{enumi})}
\setcounter{enumi}{1}
\tightlist
\item
  Write down the set of outcomes corresponding to each of the following
  events\\
\end{enumerate}

\begin{itemize}
\tightlist
\item
  A: ``We get exactly one tails''\\
\item
  B: ``The coin always comes with the same side up''\\
  \emph{Solution}:\\
\item
  A: \(\{(H,H,H,T), (H,H,T,H), (H,T,H,H), (T,H,H,H)\}\)\\
\item
  B: \(\{(H,H,H,H), (T,T,T,T)\}\)
\end{itemize}

\begin{enumerate}
\def\labelenumi{(\alph{enumi})}
\setcounter{enumi}{2}
\item
  Summarise in words the meaning of the event \(A\cup B\).\\
  \emph{Solution}:\\
  \(A\up B\) is the union of events \(A\) and \(B\). The resulting event
  occurs \emph{either} \(A\) or \(B\) occurs. (e.g.~logical ``or''
  operator).
\item
  Compute the probability for the event \(C=(A\cup B)^C\).\\
  \emph{Solution}:\\
  I first union the two sets of events:\\
  \[
  \begin{aligned}
  A\cup B&=\{(H,H,H,T), (H,H,T,H), (H,T,H,H), (T,H,H,H)\}\cup\{(H,H,H,H), (T,T,T,T)\} \\
  &=\{(H,H,H,T), (H,H,T,H), (H,T,H,H), (T,H,H,H), (H,H,H,H), (T,T,T,T)\}
  \end{aligned}
  \] The probability of this occurring is
  \(\frac{|A\cup B|}{|\Omega|}\), which is \(\frac{3}{8}\). The
  complement of this is \(1-\frac{3}{8}=\frac{5}{8}\), so
  \(P((A\cup B)^c)=\frac{5}{8}\).
\end{enumerate}

\hypertarget{expectation-variance-discrete-distributions}{%
\section{2. Expectation, Variance, Discrete
Distributions}\label{expectation-variance-discrete-distributions}}

Let us consider the experiment of independently throwing two fair dice
characterised by the discrete random variables \(X\) and \(Y\),
respectively. The discrete random variables \(X\) and \(Y\) take the
values \(a=1,2,3,\dots,6\) and \(b=1,2,3,\dots,6\), respectively.\\
(a) Compute the expected value for the product \(Z=XY\).\\
\emph{Solution}:\\
The expected value of the product of two independent random variables is
simply the product of their individual expectations. Since they have the
same expectation, as they have the same distribution,
\(E[XY]=(E[X])^2\).\\
This expectation is \[
\begin{aligned}
E[X]&=\frac{1}{6}+\frac{2}{6}+\frac{3}{6}+\frac{4}{6}+\frac{5}{6}+\frac{6}{6} \\
&=\frac{1+2+3+4+5+6}{6}=\frac{21}{6}=3.5
\end{aligned}
\] Therefore, \[
E[Z]=\left(\frac{21}{6}\right)^2=\frac{21^2}{6^2}=\frac{441}{36}=12.25
\]

\begin{enumerate}
\def\labelenumi{(\alph{enumi})}
\setcounter{enumi}{1}
\tightlist
\item
  Write down the probability mass function for the the (sic) discrete
  random variable \(Z\) defined by the product \(Z=XY\).\\
  \emph{Solution}:\\
  Where \(p_X(a)=P(X=a)\), \(p_Z(a)=P(Z=a)=P(XY=a)\). We can calculate
  this by taking all possible values for \(X\) and \(Y\):\\
\end{enumerate}

\begin{verbatim}
         X
  | X*Y | 1 | 2  | 3  | 4  | 5  | 6  |
  |  1  | 1 | 2  | 3  | 4  | 5  | 6  |
  |  2  | 2 | 4  | 6  | 8  | 10 | 12 |
Y |  3  | 3 | 6  | 9  | 12 | 15 | 18 |
  |  4  | 4 | 8  | 12 | 16 | 20 | 24 |
  |  5  | 5 | 10 | 15 | 20 | 25 | 30 |
  |  6  | 6 | 12 | 18 | 24 | 30 | 36 |
\end{verbatim}

Then we can get the probabilities of each value based on how many times
it shows up:

\begin{verbatim}
| a    |  1   |  2   |  3   |  4   |  5   |  6   |  8   |  9   |  10  |
| p(a) | 1/36 | 2/36 | 2/36 | 3/36 | 2/36 | 4/36 | 2/36 | 1/36 | 2/36 |
| a    |  12  |  15  |  16  |  18  |  20  |  24  |  25  |  30  |  36  |
| p(a) | 4/36 | 2/36 | 1/36 | 2/36 | 2/36 | 2/36 | 1/36 | 2/36 | 1/36 |
\end{verbatim}

\begin{enumerate}
\def\labelenumi{(\alph{enumi})}
\setcounter{enumi}{2}
\tightlist
\item
  Compute the variance of the product \(Z=XY\).\\
  \emph{Solution}:\\
  One of the formulas for variance is \(Var(Z)=E[Z^2]-(E[Z])^2\). Part
  of this can be filled out from the previous calculation.\\
  \[
  Var(Z)=E[Z^2]-(E[Z])^2=E[Z^2]-(12.25)^2=E[Z^2]-150.0625.
  \] \(E[Z^2]\) can be calculated using the change of variable formula,
  which states \[
  E[g(X)]=\sum_i g(a_i)P(X=a_i)
  \] where each \(a_i\) is a value that \(X\) takes.\\
  So, \[
  \begin{aligned}
  E[Z^2]&=\sum_i a_i^2P(Z=a_i) \\
  &=1^2\cdot\frac{1}{36}+2^2\cdot\frac{2}{36}+\dots+30^2\cdot\frac{2}{36}+36^2\cdot\frac{1}{36} \\
  &=230.02\bar{7}
  \end{aligned}
  \] This can be inserted into the equation, and we get the variance: \[
  Var(Z)=E[Z^2]-150.0625=230.02\bar{7}-150.0625=79.9652\bar{7}
  \] So \(Var(Z)\approx80\).
\end{enumerate}

\hypertarget{maximum-likelihood}{%
\section{3. Maximum likelihood}\label{maximum-likelihood}}

Let \(x_1,x_2,\dots,x_n\) be a dataset that is a realisation of a random
sample from a \(U(\alpha,\beta)\) distribution, where \(\alpha\) and
\(\beta\) are the unknown parameters.\\
(a) Write down the likelihood function of the parameters.\\
\emph{Solution}:\\
Since the distribution has two unknown variables, this will be a
likelihood function of multiple parameters. The likelihood function is
defined by \[
L(\alpha,\beta)=f_{(\alpha,\beta)}(x_1)\dots f_{(\alpha,\beta)}(x_n)
\] where each \(f_{(\alpha,\beta)}(x)\) is the \(U(\alpha,\beta)\) PDF:
\[
f_{(\alpha,\beta)}(x)=\frac{1}{\beta-\alpha}.
\]

\begin{enumerate}
\def\labelenumi{(\alph{enumi})}
\setcounter{enumi}{1}
\tightlist
\item
  Determine the maximum likelihood estimates for the parameters
  \(\alpha\) and \(\beta\).\\
  \emph{Solution}:\\
  Since \(\ln(f_{(\alpha,\beta)}(x))=-\ln(\beta-\alpha)\), one finds for
  the loglikelihood that \[
  \ell(\alpha,\beta)=-n\ln(\beta-\alpha)
  \] The partial derivatives of \(\ell\) are \[
  \begin{aligned}
  \frac{\partial\ell}{\partial\alpha}&=\frac{n}{\beta-\alpha} \\
  \frac{\partial\ell}{\partial\beta}&=-\frac{n}{\beta-\alpha}
  \end{aligned}
  \] Solving for \(\frac{\partial\ell}{\partial\alpha}=0\) and
  \(\frac{\partial\ell}{\partial\beta}=0\) yields \[
  \text{it doesnt work :(}
  \]
\end{enumerate}

\hypertarget{small-r-problems}{%
\section{4. Small R Problems}\label{small-r-problems}}

\begin{enumerate}
\def\labelenumi{(\alph{enumi})}
\tightlist
\item
  The data set \texttt{firstchi} (\texttt{UsingR}) contains the age of
  the mother at birth of the first child. Investigate the data set by
  computing several simple statistics on this data set. Summarise your
  findings.\\
  \emph{Solution}:
\end{enumerate}

\begin{verbatim}
## Warning: pakke 'UsingR' blev bygget under R version 4.2.3
\end{verbatim}

\begin{verbatim}
## Indlæser krævet pakke: MASS
\end{verbatim}

\begin{verbatim}
## Indlæser krævet pakke: HistData
\end{verbatim}

\begin{verbatim}
## Indlæser krævet pakke: Hmisc
\end{verbatim}

\begin{verbatim}
## Indlæser krævet pakke: lattice
\end{verbatim}

\begin{verbatim}
## Indlæser krævet pakke: survival
\end{verbatim}

\begin{verbatim}
## Indlæser krævet pakke: Formula
\end{verbatim}

\begin{verbatim}
## Indlæser krævet pakke: ggplot2
\end{verbatim}

\begin{verbatim}
## 
## Vedhæfter pakke: 'Hmisc'
\end{verbatim}

\begin{verbatim}
## De følgende objekter er maskerede fra 'package:base':
## 
##     format.pval, units
\end{verbatim}

\begin{verbatim}
## 
## Vedhæfter pakke: 'UsingR'
\end{verbatim}

\begin{verbatim}
## Det følgende objekt er maskeret fra 'package:survival':
## 
##     cancer
\end{verbatim}

\begin{verbatim}
##    Min. 1st Qu.  Median    Mean 3rd Qu.    Max. 
##   14.00   20.00   23.00   23.98   26.00   42.00
\end{verbatim}

\begin{verbatim}
## [1] 6.254258
\end{verbatim}

As the mean is slightly higher than the median, the distribution is
slightly right-leaning. We can also see from the standard deviation that
roughly 68\% of the women were between the ages of 17 and 30, assuming
normality.

\begin{enumerate}
\def\labelenumi{(\alph{enumi})}
\setcounter{enumi}{1}
\tightlist
\item
  The data set \texttt{rat} (\texttt{UsingR}) contains the survival
  times of 20 rats exposed to radiation. Visualise the data in
  appropriate means and discuss in the light of the data and your
  knowledge, what kind of parametric model would you choose for the
  distribution of the survival times.\\
  \emph{Solution}:
\end{enumerate}

\begin{Shaded}
\begin{Highlighting}[]
\FunctionTok{plot}\NormalTok{(}\FunctionTok{density}\NormalTok{(rat))}
\end{Highlighting}
\end{Shaded}

\includegraphics{Solutions_files/figure-latex/unnamed-chunk-2-1.pdf}
This distribution vaguely resembles the normal distribution, with a
right lean.

\begin{enumerate}
\def\labelenumi{(\alph{enumi})}
\setcounter{enumi}{2}
\tightlist
\item
  Consider the dataset \texttt{kid.weights} (\texttt{UsingR}), that
  reports information about a sample of 250 kids. Select the kids up
  until 9 years old (i.e.~with an age strictly less than 108 months).
  Plot a scatter plot of the weight versus the height. Compute a linear
  regression model and add the regression line to the plot. What
  conclusions can you derive?\\
  \emph{Solution}:
\end{enumerate}

\begin{Shaded}
\begin{Highlighting}[]
\NormalTok{youngins }\OtherTok{\textless{}{-}}\NormalTok{ kid.weights[kid.weights}\SpecialCharTok{$}\NormalTok{age }\SpecialCharTok{\textless{}} \DecValTok{108}\NormalTok{,]}
\NormalTok{cool\_line }\OtherTok{\textless{}{-}} \FunctionTok{lm}\NormalTok{(youngins}\SpecialCharTok{$}\NormalTok{height }\SpecialCharTok{\textasciitilde{}}\NormalTok{ youngins}\SpecialCharTok{$}\NormalTok{weight)}

\FunctionTok{plot}\NormalTok{(youngins}\SpecialCharTok{$}\NormalTok{weight, youngins}\SpecialCharTok{$}\NormalTok{height)}
\FunctionTok{abline}\NormalTok{(cool\_line)}
\end{Highlighting}
\end{Shaded}

\includegraphics{Solutions_files/figure-latex/unnamed-chunk-3-1.pdf}

\begin{Shaded}
\begin{Highlighting}[]
\NormalTok{cool\_line}
\end{Highlighting}
\end{Shaded}

\begin{verbatim}
## 
## Call:
## lm(formula = youngins$height ~ youngins$weight)
## 
## Coefficients:
##     (Intercept)  youngins$weight  
##         21.1987           0.4007
\end{verbatim}

Without doing any statistical tests, the plot seems to show a pretty
strong correlation between weight and height. Based on the knowledge of
the data though, it is probably not weight that causes height, but
rather age that causes both.

\begin{enumerate}
\def\labelenumi{(\alph{enumi})}
\setcounter{enumi}{3}
\tightlist
\item
  Assume that you have implemented a scientific method that you compare
  to the state-of-the-art published elsewhere. You use the evaluation
  metric \(G\) where a bigger value refers to a better outcome. Using
  independent experiments you get five scores for the state-of-the-art
  (0.908,0.915,0.908,0.905,0.904) and five for your method
  (0.910,0.914,0.909,0.914,0.910). State the null and alternative
  hypothesis and test if there is statistical evidence that your method
  is better than the state-of-the-art. \emph{Solution}:\\
  The null hypothesis is that the methods are equal, and the alternate
  is that they are not.
\end{enumerate}

\begin{Shaded}
\begin{Highlighting}[]
\FunctionTok{t.test}\NormalTok{(}\FunctionTok{c}\NormalTok{(}\FloatTok{0.908}\NormalTok{, }\FloatTok{0.915}\NormalTok{, }\FloatTok{0.908}\NormalTok{, }\FloatTok{0.905}\NormalTok{, }\FloatTok{0.904}\NormalTok{), }\FunctionTok{c}\NormalTok{(}\FloatTok{0.910}\NormalTok{, }\FloatTok{0.914}\NormalTok{, }\FloatTok{0.909}\NormalTok{, }\FloatTok{0.914}\NormalTok{, }\FloatTok{0.910}\NormalTok{), }\AttributeTok{alternative=}\StringTok{"two.sided"}\NormalTok{)}
\end{Highlighting}
\end{Shaded}

\begin{verbatim}
## 
##  Welch Two Sample t-test
## 
## data:  c(0.908, 0.915, 0.908, 0.905, 0.904) and c(0.91, 0.914, 0.909, 0.914, 0.91)
## t = -1.5423, df = 6.2836, p-value = 0.1717
## alternative hypothesis: true difference in means is not equal to 0
## 95 percent confidence interval:
##  -0.008735838  0.001935838
## sample estimates:
## mean of x mean of y 
##    0.9080    0.9114
\end{verbatim}

p is not small enough to reject null hypothesis, our model is probably
just as good as the state-of-the-art.

\end{document}
